%-----------------------------------------------------------------------------------------------------------------------------------------------%
%	The MIT License (MIT)
%
%	Copyright (c) 2021 Jitin Nair
%
%	Permission is hereby granted, free of charge, to any person obtaining a copy
%	of this software and associated documentation files (the "Software"), to deal
%	in the Software without restriction, including without limitation the rights
%	to use, copy, modify, merge, publish, distribute, sublicense, and/or sell
%	copies of the Software, and to permit persons to whom the Software is
%	furnished to do so, subject to the following conditions:
%	
%	THE SOFTWARE IS PROVIDED "AS IS", WITHOUT WARRANTY OF ANY KIND, EXPRESS OR
%	IMPLIED, INCLUDING BUT NOT LIMITED TO THE WARRANTIES OF MERCHANTABILITY,
%	FITNESS FOR A PARTICULAR PURPOSE AND NONINFRINGEMENT. IN NO EVENT SHALL THE
%	AUTHORS OR COPYRIGHT HOLDERS BE LIABLE FOR ANY CLAIM, DAMAGES OR OTHER
%	LIABILITY, WHETHER IN AN ACTION OF CONTRACT, TORT OR OTHERWISE, ARISING FROM,
%	OUT OF OR IN CONNECTION WITH THE SOFTWARE OR THE USE OR OTHER DEALINGS IN
%	THE SOFTWARE.
%	
%
%-----------------------------------------------------------------------------------------------------------------------------------------------%

%----------------------------------------------------------------------------------------
%	DOCUMENT DEFINITION
%----------------------------------------------------------------------------------------

% article class because we want to fully customize the page and not use a cv template
%\RequirePackage{plautopatch}
\documentclass[uplatex,dvipdfmx,a4paper,11pt]{jsarticle}
%\documentclass[a4paper,11pt]{article}

%----------------------------------------------------------------------------------------
%	FONT
%----------------------------------------------------------------------------------------

% % fontspec allows you to use TTF/OTF fonts directly
% \usepackage{fontspec}
% \defaultfontfeatures{Ligatures=TeX}

% % modified for ShareLaTeX use
% \setmainfont[
% SmallCapsFont = Fontin-SmallCaps.otf,
% BoldFont = Fontin-Bold.otf,
% ItalicFont = Fontin-Italic.otf
% ]
% {Fontin.otf}

%----------------------------------------------------------------------------------------
%	PACKAGES
%----------------------------------------------------------------------------------------
\usepackage{url}
\usepackage{parskip} 	
\usepackage[normalem]{ulem}

%other packages for formatting
\RequirePackage{color}
\RequirePackage{graphicx}
\usepackage[usenames,dvipsnames]{xcolor}
\usepackage[scale=0.9]{geometry}

%tabularx environment
\usepackage{tabularx}

%for lists within experience section
\usepackage{enumitem}

% centered version of 'X' col. type
\newcolumntype{C}{>{\centering\arraybackslash}X} 

%to prevent spillover of tabular into next pages
\usepackage{supertabular}
\usepackage{tabularx}
\newlength{\fullcollw}
\setlength{\fullcollw}{0.47\textwidth}

%custom \section
\usepackage{titlesec}				
\usepackage{multicol}
\usepackage{multirow}

%CV Sections inspired by: 
%http://stefano.italians.nl/archives/26
\titleformat{\section}{\Large\scshape\raggedright}{}{0em}{}[\titlerule]
\titlespacing{\section}{0pt}{10pt}{10pt}

%for publications
\usepackage[style=authoryear,sorting=ynt, maxbibnames=2]{biblatex}

%Setup hyperref package, and colours for links
\usepackage[unicode, draft=false]{hyperref}
\definecolor{blue(pigment)}{rgb}{0.2, 0.2, 0.6}
\definecolor{linkcolour}{rgb}{0,0.2,0.6}
\hypersetup{colorlinks,breaklinks,urlcolor=linkcolour,linkcolor=linkcolour}
\addbibresource{citations.bib}
\setlength\bibitemsep{1em}

%for social icons
\usepackage{fontawesome5}

%debug page outer frames
%\usepackage{showframe}

%----------------------------------------------------------------------------------------
%	BEGIN DOCUMENT
%----------------------------------------------------------------------------------------
\begin{document}

% non-numbered pages
\pagestyle{empty} 

%----------------------------------------------------------------------------------------
%	TITLE
%----------------------------------------------------------------------------------------

% \begin{tabularx}{\linewidth}{ @{}X X@{} }
% \huge{Your Name}\vspace{2pt} & \hfill \emoji{incoming-envelope} email@email.com \\
% \raisebox{-0.05\height}\faGithub\ username \ | \
% \raisebox{-0.00\height}\faLinkedin\ username \ | \ \raisebox{-0.05\height}\faGlobe \ mysite.com  & \hfill \emoji{calling} number
% \end{tabularx}

\begin{tabularx}{\linewidth}{@{} C @{}}
\Huge{アンダーランド ジェイク} \\
Underland, Jake\\[7.5pt]
\href{https://github.com/jaked0626}{\raisebox{-0.05\height}\faGithub\ jaked0626} \ $|$ \ 
\href{https://linkedin.com/in/jake-underland-720126201/}{\raisebox{-0.05\height}\faLinkedin\ Jake Underland} \ $|$ \ 
\href{https://jaked0626.github.io/digital-biz-card/}{\raisebox{-0.05\height}\faGlobe \ JakesBizCard.com} \ $|$ \ 
\href{mailto:jakez0626@gmail.com}{\raisebox{-0.05\height}\faEnvelope \ jakez0626@gmail.com} \ $|$ \ 
\href{tel:+819014618440}{\raisebox{-0.05\height}\faPhone \ 090.1461.8440} \\
\faTenge{ 114-0023 東京都北区滝野川 6-60-8}\\
\end{tabularx}

%----------------------------------------------------------------------------------------
% EXPERIENCE SECTIONS
%----------------------------------------------------------------------------------------

%Interests/ Keywords/ Summary
%\section{Summary}
%This CV can also be automatically complied and published using GitHub Actions. For details, \href{https://github.com/jitinnair1/autoCV}{click here}.

%Experience
\section{職歴 - Work Experience}

{\bf \href{https://www.optimind.tech/}{Optimind Inc.}} { - アルゴリズムチームエンジニア兼PdM} \hfill {2022年4月--現在} 
\\- ルート最適化サービス\href{https://loogia.jp/?_ga=2.72237267.1726635224.1669133714-1962149220.1665458028}{Loogia}のアルゴリズムチームでPython, c++, JSで\emph{エンジン開発},GKEとTerraformで\emph{サーバ保守}
\\- PdMとして\textbf{バックログ・ロードマップ管理}.監視した開発例:p99の計算時間を\textbf{97%高速化}.多点間経路探索へのヒューリスティック導入の研究開発(平均計算速度\textbf{20%高速化}).データパイプラインプロダクトの\textbf{MVP要件定義},リリース.
%Customer User JourneyからSLOの定義,\textbf{SLIの実装}を先導.\textbf{インシデント対応の体制整備}.データパイプラインプロダクトのPdMとして\textbf{ミッション・ビジョンの定義},\textbf{MVP開発}とリリースまで監督.
%\\- 監視した開発例:p99のレスポンス計算速度(ネットワークレイテンシを除く)を\textbf{97%高速化}.多点間経路探索ダイクストラへのヒューリスティック導入の研究開発(平均計算速度\textbf{20%高速化}).

{ \bf\textcolor{blue(pigment)}{早稲田大学政治経済学部}} { - リサーチアシスタント} \hfill {2022年2月--2022年8月} 
\\- 開発途上国における電力使用のデータ分析.Rで線形モデル,IVモデル,GMMを活用した動学構造推定

{\bf\textcolor{blue(pigment)}{日本銀行}}{ - IT 夏季インターン} \hfill { 2021年8月} 

{\bf \href{https://www.morganstanley.co.jp/ja}{モルガン・スタンレーMUFG証券}}{ - IT 夏季インターン} \hfill {2021年8月} 
\\- 社内アプリのログを監視し,システムの異常を検知するツールをPython, FastAPI, DataDogで作成

{ \bf\href{https://gdsc.community.dev/waseda-university/}{Google Student Club}} { - Backend チームリーダー} \hfill {2021年6月--2022年6月} 
\\- 日本で最大のGoogle学生組織である早稲田大学でバックエンドチームをリード.Golangによるマイクロサービス開発.
\\- ソフトウェアを競う大会 Solution Challenge で世界トップ50入り.

{\bf \href{https://createbase.work/}{CreateBase}} { - フルスタックデベロッパー} \hfill {2021年6月--2022年4月} 
\\- Python, Django, BeautifulSoup, Seleniumにより,データをクロールして可視化するダッシュボードのフルスタック開発.

%----------------------------------------------------------------------------------------
%	EDUCATION
%----------------------------------------------------------------------------------------
\section{学歴 - Education}
\begin{tabularx}{\linewidth}{@{}l X@{}}	
2019 - 2023 & \uline{早稲田大学} 政治経済学部国際政治経済学科 \hfill \normalsize (GPA: 3.7/4.0) \\
&  \href{https://tadaohoshino.wordpress.com/}{星野匡郎}ゼミで機械学習理論,\href{https://waseda.pure.elsevier.com/en/persons/hisatoshi-tanaka}{田中久俊}ゼミで計量経済理論を研究.\\
& 活動: \href{http://www2.cie-waseda.jp/glfp/jp/about/program.html}{グローバルリーダーシッププログラム},\href{https://gdsc.community.dev/waseda-university/}{Google Student Club バックエンドリーダー},計量経済学RA\\
2020 - 2021 & \uline{シカゴ大学} 交換留学 \hfill (GPA: 3.93/4.0) \\
& 活動: 公共経済学RA,教育経済学RA
\end{tabularx}

%Projects
% \section{作品 - Projects}

% \begin{tabularx}{\linewidth}{ @{}l r@{} }
% \textbf{Some Project} & \hfill \href{https://some-link.com}{Link to Demo} \\[3.75pt]
% \multicolumn{2}{@{}X@{}}{long long line of blah blah that will wrap when the table fills the column width long long line of blah blah that will wrap when the table fills the column width long long line of blah blah that will wrap when the table fills the column width long long line of blah blah that will wrap when the table fills the column width}  \\
% \end{tabularx}

%----------------------------------------------------------------------------------------
%	PUBLICATIONS
%----------------------------------------------------------------------------------------
%\section{Publications}
%\begin{refsection}[citations.bib]
%\nocite{*}
%\printbibliography[heading=none]
%\end{refsection}


%----------------------------------------------------------------------------------------
%	SKILLS
%----------------------------------------------------------------------------------------
\section{スキル - Skills}
\begin{tabular}{ @{} >{\bfseries}l @{\hspace{6ex}} l }
言語: \ & 日本語 (ネイティブ), 英語 (ネイティブ) \\
プログラミング: \ & Python, Golang,  SQL (Postgres), C++, Javascript, R \\
& FastAPI, Django, Gin, React.js, Next.js\\
& Pandas, NumPy, Matplotlib, Sklearn \\
& Docker, AWS, GCP
\end{tabular}


%----------------------------------------------------------------------------------------
%	CERTIFICATES
%----------------------------------------------------------------------------------------
\section{資格等 - Certificates}
- 日本統計学検定3級, 最優秀賞, 2019 \\
- 早稲田大学新井孝晋奨学金

\vfill
\center{\footnotesize 最終更新日: \today}

\end{document}